%!TEX program = xelatex
\documentclass[a4paper,10pt]{article}

\usepackage[table]{xcolor}
\usepackage[ruled]{algorithm2e}

\usepackage{graphicx}
\usepackage{fullpage}
\usepackage{amsmath,boxedminipage}
\usepackage{amssymb,amsthm}
\usepackage[totalwidth=166mm,totalheight=240mm]{geometry}
\usepackage{framed}
\usepackage{hyperref}

%-- coding: UTF-8 --
\usepackage{ctex}

\newtheorem{theorem}{Theorem}[section]
\newtheorem{lemma}[theorem]{Lemma}

\parindent0mm
%\pagestyle{empty}


\usepackage{tikz}
\usetikzlibrary{arrows}


\newcommand{\nop}[1]{}

\newcounter{aufgc}
\newenvironment{exercise}[1]%
{\refstepcounter{aufgc}\textbf{Exercise \arabic{aufgc}} \emph{#1}\\}
{
	
	\hrulefill\medskip}%

\renewcommand{\labelenumi}{(\alph{enumi})}

\providecommand{\abs}[1]{\lvert#1\rvert} \providecommand{\norm}[1]{\lVert#1\rVert}
\providecommand{\vt}[1]{\mathbf{\mathrm{#1}}}

\newcommand{\E}{\mathrm{E}}

\newcommand{\Var}{\mathrm{Var}}
\newcommand{\rank}{\mathrm{rank}}
\begin{document}
	
	\vspace{-2em}
	\begin{minipage}[b]{0.58\textwidth}
		\large 
中国科学技术大学\\
计算机科学与技术学院
	\end{minipage}
	
	\hrulefill
	
	%\vspace{0.2cm}
	\begin{center}
		{\large \bf 《大数据算法》作业\\[0.5mm]
			2022年春}\\
		\textcolor{red}{截止日期:2022年6月9日23:59}
		% \bigskip
		
	\end{center}
	%\vspace{0.1cm}
	
	
	
%注:在下面习题中,[BHK]指的是参考文献``Foundations of Data Science'' \url{https://www.cs.cornell.edu/jeh/book.pdf}。	
	
	
	
	
	
	%\newpage
	\hrulefill\medskip
	
	\begin{exercise}{20分}
证明(关于欧氏$k$-means问题的)coresets满足下面的可组合性质(composability):

令$A_1,A_2\subseteq \mathbb{R}^d$是两个互不相交的集合。假设集合$S_1$及权重函数$w_1:S_1\to \mathbb{R}$和集合$S_2$及权重函数$w_2:S_2\to \mathbb{R}$分别是$A_1$和$A_2$的$(k,\varepsilon)$-coresets。那么$S_1\cup S_2$及函数$w_1+w_2: S_1\cup S_2\to \mathbb{R}$是$A_1\cup A_2$的$(k,\varepsilon)$-coreset。

\textbf{注:} 这里$w_1+w_2$的定义如下:
\begin{equation*}
(w_1+w_2)(x)=\begin{cases}
		w_1(x) & \text{如果$x\in S_1\setminus S_2$},\\
			w_2(x) & \text{如果$x\in S_2\setminus S_1$},\\
		w_1(x)+w_2(x) & \text{如果$x\in S_2\cap S_1$}
	\end{cases}
\end{equation*}
\end{exercise}
	
	\begin{exercise}{20分}
\begin{itemize}
\item 对于欧氏$k$-median问题,我们可以限制$k$个中心点$c_1,\dots,c_k$都是来自于输入数据集$A$中的,也可以允许它们是来自整个欧氏空间$\mathbb{R}^d$的。证明在这两种情况下,问题的最优解所对应的目标函数值的比值不超过$2$。
\item 对于欧氏$k$-means问题,我们可以限制$k$个中心点$c_1,\dots,c_k$都是来自于输入数据集$A$中的,也可以允许它们是来自整个欧氏空间$\mathbb{R}^d$的。证明在这两种情况下,问题的最优解所对应的目标函数值的比值不超过$4$。
\end{itemize}

	\end{exercise}
	
	%\hrulefill
	
	\vspace{0.2cm}
	\begin{exercise}{20分}
考虑平面$\mathbb{R}^2$上的$k$-median问题,其中我们要求$k$个中心点$c_1,\dots,c_k$都是来自于输入数据集$A$中的。考虑枚举所有可能的聚类并从中选出具有最小代价的聚类。我们可以将所有的$n$个点进行标号,每个标号是$\{1,\dots,k\}$中的一个数。注意到所有可能的标号数是$k^n$,这对应着高昂的时间。

证明我们可以在$n^{O(k)}$时间内找到最优的聚类。
	\end{exercise}
	
	\vspace{0.2cm}
		\begin{exercise}{10分}
		令$a,b,c$为任意三个实数。证明对于任意的$\varepsilon\in (0,1)$,下面的不等式(即推广的三角不等式)成立:
		\[
		\abs{\,\abs{a-c}^2-\abs{b-c}^2}\leq \frac{12}{\varepsilon}\cdot \abs{a-b}^2 +2\varepsilon\cdot \abs{a-c}^2
		\]
		
	\end{exercise}
\vspace{0.2cm}

	\begin{exercise}{30分}
考虑$k$-means问题。令$A=\{a_1,\dots,a_n\}\subseteq \mathbb{R}^d$为一个含有$n$个点的集合。对于$A$的任意一个$k$-划分$C_1,\cdots,C_k$,定义
\[
D(A,\{C_i\}_{i=1,\dots,k}):=\sum_{i=1}^k\sum_{a\in C_i} \norm{a-\mu(C_i)}^2,
\]
这里的$\mu(C_i)=\frac{1}{|C_i|}\sum_{a\in C_i}a$。


令$\varepsilon\in (0,1)$。令$k'\geq \Omega(\frac{\log n}{\varepsilon^2})$为JL引理(Johnson-Lindenstrauss Lemma)中将$A$中的点通过随机投影降维之后的维度。

证明存在一个线性映射$f:\mathbb{R}^d\to \mathbb{R}^{k'}$满足对于$A$的所有的$k$-划分$C_1,\cdots,C_k$,下面的式子成立:
\[
\abs{D(A,\{C_i\}_{i=1,\dots,k}) - D(f(A),\{f(C_i)\}_{i=1,\dots,k})}\leq \varepsilon\cdot  D(A,\{C_i\}_{i=1,\dots,k}),
\]
这里$f(C_i)=\{f(x)\mid x\in C_i\}$,$f(A)=\{f(x)\mid x\in A\}$。这里的$f$是$A$与$f(A)$之间的双射。	
	\end{exercise}
	

	

	

	
\end{document}

%%% Local Variables:ƒ
%%% mode: latex
%%% TeX-master: t
%%% End:
