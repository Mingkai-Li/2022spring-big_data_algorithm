%!TEX program = xelatex
\documentclass[a4paper,10pt]{article}

\usepackage[table]{xcolor}
\usepackage[ruled]{algorithm2e}

\usepackage{graphicx}
\usepackage{fullpage}
\usepackage{amsmath,boxedminipage}
\usepackage{amssymb,amsthm}
\usepackage[totalwidth=166mm,totalheight=240mm]{geometry}
\usepackage{framed}
\usepackage{hyperref}

%-- coding: UTF-8 --
\usepackage{ctex}

\newtheorem{theorem}{Theorem}[section]
\newtheorem{lemma}[theorem]{Lemma}

\parindent0mm
%\pagestyle{empty}


\usepackage{tikz}
\usetikzlibrary{arrows}


\newcommand{\nop}[1]{}

\newcounter{aufgc}
\newenvironment{exercise}[1]%
{\refstepcounter{aufgc}\textbf{Exercise \arabic{aufgc}} \emph{#1}\\}
{
	
	\hrulefill\medskip}%

\renewcommand{\labelenumi}{(\alph{enumi})}

\providecommand{\abs}[1]{\lvert#1\rvert} \providecommand{\norm}[1]{\lVert#1\rVert}
\providecommand{\vt}[1]{\mathbf{\mathrm{#1}}}

\newcommand{\E}{\mathrm{E}}

\newcommand{\Var}{\mathrm{Var}}
\newcommand{\rank}{\mathrm{rank}}
\begin{document}
	
	\vspace{-2em}
	\begin{minipage}[b]{0.58\textwidth}
		\large 
中国科学技术大学\\
计算机科学与技术学院
	\end{minipage}
	
	\hrulefill
	
	%\vspace{0.2cm}
	\begin{center}
		{\large \bf 《大数据算法》作业\\[0.5mm]
			2022年春}\\
		\textcolor{red}{截止日期:2022年5月6日23:59}
		% \bigskip
		
	\end{center}
	%\vspace{0.1cm}
	
	
	
%注:在下面习题中,[BHK]指的是参考文献``Foundations of Data Science'' \url{https://www.cs.cornell.edu/jeh/book.pdf}。	
	
	
	
	
	
	%\newpage
	\hrulefill\medskip
	
	\begin{exercise}{20分}
	在\textsc{CountSketch}算法及其分析中,我们证明了如果选择$w>3k^2$,$d=\Omega(\log n)$, 那么以$1-\frac{1}{n}$的概率,对于任意$i\in [n]$,$\abs{\tilde{x}_i-x_i}\leq \frac{\norm{x}_2}{k}$。这个估计有可能在某些情况是比较坏的,例如当$\norm{x}_2$的值主要集中在少数几个坐标上的时候。
	
	
	对于固定的整数$\ell>0$,对于任意$i\in [n]$,定义向量$y^{(i)}\in \mathbb{R}^n$如下:
	\begin{equation*}
		y^{(i)}_j=\begin{cases}
			0 & \text{如果$j=i$或者$j$是$x$中(在绝对值意义下)最大的$\ell$个值所对应的坐标之一},\\
			x_j & \text{否则}
		\end{cases}
	\end{equation*}
	证明对于$\ell=k^2$,如果$w=6k^2$,$d=\Omega(\log n)$,那么以$1-\frac{1}{n}$的概率,对于任意$i\in [n]$,$\abs{\tilde{x}_i-x_i}\leq \frac{\norm{y^{(i)}}_2}{k}$。
	
\end{exercise}
	
	\begin{exercise}{20分}
假设$k_1,k_2$是两个核(kernel)函数。证明:
\begin{enumerate}
\item 对于任意常数$c\geq 0$,$c k_1$是一个核函数。
\item 对于任意标量(scalar)函数 $f$,$k_3(\vt{x},\vt{y})=f(\vt{x})f(\vt{y})\cdot k_1(\vt{x},\vt{y})$是一个核函数。
\item $k_1+k_2$是一个核函数。
\item $k_1\cdot k_2$是一个核函数。
\end{enumerate}
	\end{exercise}
	
	%\hrulefill
	
	\vspace{0.2cm}
	\begin{exercise}{20分}
令$X=\mathbb{R}^d$, 并定义$\mathcal{H}$为$X$上的所有axis-parallel boxes所构成的集合。具体来说,$\mathcal{H}=\{h_{\vt{a},\vt{b}}\mid \vt{a},\vt{b}\in X\}$。对于$\vt{x}\in X$,$h_{\vt{a},\vt{b}}(\vt{x})$定义如下:

\begin{equation*}
	h_{\vt{a},\vt{b}}(\vt{x})=\begin{cases}
		1 & \text{如果$a_i\leq x_i\leq b_i$对于任意的$i=1,\dots,d$},\\
		-1 & \text{否则。}
	\end{cases}
\end{equation*}
选择一个可以被$\mathcal{H}$打散(shatter)的点集$V$,并
\begin{enumerate}
\item 通过证明$V$是可以被$\mathcal{H}$打散的,来证明$\mathcal{H}$的VC-维(VC-dimension)至少为$|V|$;
\item 通过证明不存在大小为$|V|+1$的点集是可以被$\mathcal{H}$打散的,来证明$\mathcal{H}$的VC-维至多为$|V|$。
\end{enumerate}
	\end{exercise}
	
	\vspace{0.2cm}
	\begin{exercise}{20分}
一个点集$S\subseteq \mathbb{R}^d$被称为是``可以被一个间隔(margin)为$\gamma$的线性分割子(linear  separator)所打散的'',如果对于$S$中所有点的任意一个分类标号(labelling)都是可以被某个间隔为$\gamma$的线性分割子来实现的。证明在单位球中,不存在一个大小为$\frac{1}{\gamma^2}+1$且可以被一个间隔为$\gamma$的线性分割子所打散的集合。

\textbf{提示:} 考虑感知机(Perceptron)算法;尝试反证法。
	\end{exercise}
	
	\begin{exercise}{20分}
令实例空间(instance space)$X=\{0,1\}^d$,并令$\mathcal{H}$为所有的3-合取范式公式(3-CNF formula)所构成的类。具体来说,考虑所有的由至多3个文字(literal)的析取(即OR)所构成的逻辑子句(clause),$\mathcal{H}$是所有的可以被描述成这样的子句的合取(conjunction)形式的概念(concepts)构成的集合。例如,目标概念$c^*$可能为$(x_1 \lor \bar{x_2}\lor x_3)\land (x_2\lor x_4)\land (\bar{x}_1\lor x_3)\land (x_2\lor x_3\lor x_4)$。假设我们在PAC-learning的设定中:训练数据中的样本(examples)是根据某个分布$D$抽样出来的,它们是根据某个3-合取范式公式$c^*$来被标号的。
\begin{enumerate}
\item 给出样本个数m的一个下界,保证以至少$1-\delta$的概率,对于所有的与训练数据一致(consistent)的3-合取范式公式,其错误都不超过$\varepsilon$,这里的错误是相对应于分布$D$而言的。
\item 假设存在一个3-合取范式公式与训练数据一致,给出一个多项式时间的算法来找到一个这样的公式。
\end{enumerate}
	\end{exercise}
	

	

	
\end{document}

%%% Local Variables:ƒ
%%% mode: latex
%%% TeX-master: t
%%% End:
